\chapter{Conclusioni e Sviluppi futuri}
In conclusione, il lavoro svolto in questa tesi rappresenta un punto di partenza solido per l'ottimizzazione dei sistemi di rilevamento degli attacchi basati su dati simulati. Tuttavia, esistono molteplici direzioni per futuri sviluppi. La costante evoluzione del panorama delle minacce e delle tecnologie richiede un continuo aggiornamento delle tecniche e degli strumenti impiegati, e la ricerca futura potrà trarre grande vantaggio dall'integrazione di nuove tecnologie di simulazione, machine learning e data augmentation.
L'impatto della data augmentation è stato significativo, mostrando come alcuni dataset inizialmente inutilizzabili in condizioni reali potessero migliorare notevolmente grazie all'inserimento di un tool come il simulatore \textit{DDoShield}. Questa metodologia consente, in particolar modo, di affrontare situazioni in cui è difficile raccogliere dati reali sufficienti, come ad esempio nei casi in cui il sistema è nuovo o non ha ancora affrontato un numero consistente di minacce. La possibilità di simulare scenari permette di ampliare la base dati, rendendo il sistema più robusto e preparato ad affrontare scenari complessi di attacco.
Tuttavia, ci sono diversi spunti di miglioramento che potrebbero essere esplorati in studi futuri. Di seguito valutiamo alcune possibilità.

\section{Ottimizzazione delle tecniche di simulazione}
Sebbene il simulatore abbia dimostrato un grande potenziale, rimane spazio per migliorare la qualità e l'affidabilità dei dati simulati. Futuri sviluppi potrebbero concentrarsi sull'ottimizzazione del processo di simulazione, introducendo tecniche di simulazione più avanzate che generino dati ancor più realistici e che si adattino meglio alle peculiarità dei singoli scenari di attacco. Ad esempio, potrebbe essere utile esplorare l'integrazione di metodi di machine learning ed algoritmi genetici nella generazione dei dati simulati(specie se si usano protocolli), per creare scenari di attacco più dinamici e realistici. 
Potrebbero inoltre essere introdotti nuovi attacchi ricostruendo cosi nuovi scenari sempre più complessi.

\section{Integrazione con altre tecniche di Data Augmentation}
Lo sviluppo futuro potrebbe anche includere l'integrazione con altre tecniche di \textit{data augmentation}, come il \textit{data synthesis} o il \textit{GAN-based augmentation} (Generative Adversarial Networks), che potrebbero ulteriormente aumentare la varietà dei dati disponibili.\cite{GAN} Questo approccio combinato potrebbe permettere di creare dataset ancora più ricchi e diversificati, migliorando la capacità dei modelli di apprendimento di riconoscere un ampio spettro di attacchi, anche quelli meno frequenti o inediti.

\section{Valutazione dell'impatto su altri algoritmi di Machine Learning}
Un altro ambito di sviluppo futuro è l'estensione dell'analisi a ulteriori algoritmi di machine learning oltre a quelli già utilizzati (K-Means, Random Forest, CNN). Sarebbe interessante valutare l'impatto del simulatore e delle tecniche di \textit{data augmentation} su modelli più recenti, come le reti neurali ricorrenti (RNN) che potrebbero offrire performance migliori, soprattutto in scenari di attacchi altamente dinamici o con caratteristiche temporali complesse. \cite{RNN}

\section{Monitoraggio e aggiornamento continuo del simulatore}
In un contesto di sicurezza informatica, le minacce evolvono costantemente. Pertanto, sarebbe auspicabile sviluppare un sistema di monitoraggio continuo e di aggiornamento del simulatore in modo da adattarlo ai nuovi tipi di attacchi. Questo processo dinamico permetterebbe al simulatore di restare sempre aggiornato, migliorando così le capacità di previsione e reazione del sistema. L'integrazione di un framework di apprendimento continuo (\textit{online learning}) potrebbe rappresentare un'opportunità interessante per questo tipo di evoluzione.

\section{Collaborazione tra più simulazioni e framework}
Un altro aspetto interessante da considerare sarebbe la collaborazione tra simulatori differenti o l'integrazione di più framework all'interno della stessa pipeline di addestramento. Unendo simulazioni diverse, provenienti da tool come \textit{DDoShield}, con altre fonti di dati (come honeypots, reti reali attaccate e attività di penetration testing), si potrebbe ulteriormente migliorare la qualità e la copertura del dataset, ampliando le capacità di difesa del sistema.
