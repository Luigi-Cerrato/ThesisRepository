\chapter{Abstract} 


L' \textit{Internet of Things} (IoT) ha introdotto una nuova era di connettività, consentendo la comunicazione e la cooperazione tra miliardi di dispositivi fisici in vari settori, dall'industria alla domotica. Tuttavia, la rapida diffusione di questi dispositivi ha posto nuove sfide per la sicurezza informatica, poiché le reti IoT sono vulnerabili a una vasta gamma di minacce, soprattutto gli attacchi Zero-Day, ovvero attacchi che si verificano quando una vulnerabilità viene sfruttata prima che venga scoperta e riparata dai produttori o responsabili della sicurezza. In questo contesto, gli \textit{Intrusion Detection Systems} (IDS) giocano un ruolo cruciale nel monitoraggio e nella protezione di tali reti.

Questa tesi propone una strategia innovativa per ottimizzare l'addestramento degli IDS in ambienti IoT, utilizzando tecniche di \textit{Data Augmentation}. L'approccio mira ad aumentare l'efficacia degli IDS nell'identificazione di minacce sconosciute, migliorando la varietà e la qualità dei dati utilizzati per il loro addestramento. Dopo una panoramica delle principali problematiche di sicurezza nell'IoT e delle tipologie di IDS, la tesi descrive nel dettaglio la metodologia adottata, inclusi gli strumenti utilizzati per simulare e generare i dati di addestramento.
I risultati ottenuti dimostrano che l'uso della \textit{Data Augmentation} contribuisce in modo significativo a migliorare le capacità di rilevamento degli IDS, soprattutto in presenza di attacchi Zero-Day. L'implementazione di questa strategia, discussa nel contesto di reti IoT, evidenzia come l'ottimizzazione dell'addestramento possa ridurre il rischio di compromissione dei dispositivi, migliorando complessivamente la sicurezza delle infrastrutture connesse.
